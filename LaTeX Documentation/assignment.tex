\section{Assignments}
\label{sec:assignments}
\begin{enumerate}
    \item You will use the light–dependent resistor (LDR), seen in Figure 3 (a), to simulate the ignition key of the car. You have to define a threshold indicating the level of light deciding whether the key is present or absent. As expected, the car only works when the key is present; otherwise, the car is OFF.
    \item You will use a section on the DIP switch (DIP), seen in Figure 3 (b), to simulate the gearbox of your car. The car can be in 4 possible states:
    \begin{itemize}
        \item OFF: The car is OFF and nothing is working.
        \item PARK: The car is going to park. You must use 2 LEDs, seen in Figure 3 (f), that will blink (at your desired frequency) for 10 times and then they will go OFF.
        \item DRIVE: The car is moving forward. The motor (MOT), seen in Figure 3 (c), rotates clockwise at a certain speed, given by the potentiometer (POT), seen in Figure 3 (d). The more you increase POT, the higher the speed. Use the motor driver (DRV), seen in Figure 3 (j), to set the direction of the rotation.
        \item BACK: The motor (MOT) rotates counterclockwise at a single predefined speed. Use the motor driver (DRV) to set the direction of the rotation.
    \end{itemize}
    \item You will use the screen (DSP), seen in Figure 3 (e), to show some important information of the car, for example: \textit{the state of the gearbox (DIP)}, the frequency and the speed of the motor rotation, additional state of the LEDs, or any other information that you may consider important.
    \item You will use another section on the DIP switch (DIP), seen in Figure 3 (b), to simulate the controls of the directional lights. The lights can be in three different options:
    \begin{itemize}
        \item LEFT: The car is turning to the left. You should blink (at your desired frequency) the corresponding directional LED.
        \item RIGHT: The car is turning to the right. You should blink (at your desired frequency) the corresponding directional LED.
        \item NONE: The car is not turning. You should not blink any directional LED.
    \end{itemize}
    \item You will use a push button (BTN), seen in Figure 3 (i), to simulate the brake pedal. When the pedal is pressed, the car will stop. You have to use an interrupt to simulate this process.
    \item You will use another push button (BTN) to activate/deactivate the horn of the car. You will use a buzzer (BZR), seen in Figure 3 (h), as the car horn. You have to use an interrupt to simulate this process.
    \item The frequency and speed of the motor rotation can be estimated using the Hall effect sensor, seen in Figure 3 (g). You have to assemble the magnet holder, seen in Figure 3 (l), into the motor shaft, and then attach the cube magnet, seen in Figure 3 (k), into the holder. Figure 2 shows an example of the mounting schema.
\end{enumerate}
