\section{Introduction}
\label{sec:introduction}
%%%%%%%%%%%%%%%%%%%%%%%%%%%%%%%%%%%%%%%%%%%%%%%%%%%%%%%%%%%%%%%%%%%%%%%%%%%%%%%%%%%%%%%%%%%%%%%%%
%
% The following paragraph contains an example of citation. Use the command (\cite{•}) with
% the appropiate citation identifier. Do not forget to use the parenthesis.
%
%%%%%%%%%%%%%%%%%%%%%%%%%%%%%%%%%%%%%%%%%%%%%%%%%%%%%%%%%%%%%%%%%%%%%%%%%%%%%%%%%%%%%%%%%%%%%%%%%
In the early 1980's, Richard Feynman proposed that a quantum computer would be
an effective tool with which to solve problems in physics and chemistry, given
that it is exponentially costly to simulate large quantum systems with
classical computers (\cite{feynman1982simulating}).

...

...

...

%%%%%%%%%%%%%%%%%%%%%%%%%%%%%%%%%%%%%%%%%%%%%%%%%%%%%%%%%%%%%%%%%%%%%%%%%%%%%%%%%%%%%%%%%%%%%%%%%
%
% This paragraph contains an example of an internal reference. Use the command
% \ref{•} along with the identifier defined with the command \label{•} to create
% a reference to an internal section on your research paper. The label can be
%  defined in any of the .tex files that you have included.
%
%%%%%%%%%%%%%%%%%%%%%%%%%%%%%%%%%%%%%%%%%%%%%%%%%%%%%%%%%%%%%%%%%%%%%%%%%%%%%%%%%%%%%%%%%%%%%%%%%
This research paper is organized as follows. Section~\ref{sec:task} offers
.... Section~\ref{sec:processor} describes the .... Section~\ref{sec:experiments}
reports a .... Finally, in Section~\ref{sec:conclusions} we state ...
