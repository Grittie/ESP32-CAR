\section{Experiments}
\label{sec:experiments}
For the full experiment, we generate quantum circuits using the two--qubit
unitaries measured for each pair during simultaneous operation, rather than a
standard gate for all pairs ...

Finally, we benchmark qubit readout using standard dispersive measurement
~\cite{wallraff2005approaching}. Measurement errors averaged over the $\ket{0}$
and $\ket{1}$ states are shown in Table~\ref{tab:pauli_errors}. We have also
measured the error when operating all qubits simultaneously, by randomly
preparing each qubit in the $\ket{0}$ or $\ket{1}$ state and then measuring all
qubits for the probability of the correct result. We find that simultaneous
readout incurs only a modest increase in per--qubit measurement errors.

%%%%%%%%%%%%%%%%%%%%%%%%%%%%%%%%%%%%%%%%%%%%%%%%%%%%%%%%%%%%%%%%%%%%%%%%%%%%%%%%%%%%%%%%%%%%%%%%%
%
% The following paragraph contains several examples:
%
%   1.- A table. Use the environment \begin{table} ... \end{table} to insert
%       a table.
%
%   2.- Centering in the page. Use the command \centering to center horizontally
%        in the page
%
%   3.- Define and create the table. Use the environment \begin{tabular} ... 
%       \end{tabular} to construct a table. The alignment of the cells is given by
%       the character 'c' and the borders are defined by the character '|'.
%
%   4.- Horizontal line. Use the command \hline to create a horizontal line (border)
%       for the table.
%
%   5.- Table columns and rows. Each column/cell is limited by the character '&" and
%       the rows are limited by the character '\\'
%
%   6.- Change the font color. Use the command \textcolor{color}{text} to apply
%       a given color to the specified text.
%
%   7.- Create a caption for the table. Use the command \caption{•} to include an 
%       explanatory text of the table. Be as self-contained as possible.
%
%   8.- Definition of the label. Use the command \label{•} to be used/referenced in
%       a different section of your research paper.
%
%%%%%%%%%%%%%%%%%%%%%%%%%%%%%%%%%%%%%%%%%%%%%%%%%%%%%%%%%%%%%%%%%%%%%%%%%%%%%%%%%%%%%%%%%%%%%%%%%
\begin{table}[t]
  \centering
  \begin{tabular}{|c|| c | c |}
    \hline
    Average error & Isolated & Simultaneous \\
    \hline
    \hline
    Single--qubit ($e_{1}$) & 0.15\% & 0.16\% \\
    \textcolor{green}{Two--qubit ($e_{2}$)} & \textcolor{green}{0.36\%} & \textcolor{green}{0.62\%} \\
    \textcolor{blue}{Two--qubit, cycle ($e_{2c}$)} & \textcolor{blue}{0.65\%} & \textcolor{blue}{0.93\%} \\
    \textcolor{orange}{Readout ($e_{r}$)} & \textcolor{orange}{3.1\%} & \textcolor{orange}{3.8\%} \\
    \hline
  \end{tabular}
  \caption{Average (mean) values of Pauli errors (black, green, blue) and
           readout errors (orange), measured on qubits in isolation and when
           operating all qubits simultaneously.}
  \label{tab:pauli_errors}
\end{table}